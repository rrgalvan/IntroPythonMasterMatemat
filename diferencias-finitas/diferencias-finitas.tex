\documentclass[11pt,spanish,a4wide]{article}

\usepackage{babel}
\usepackage[utf8]{inputenc}
\usepackage[T1]{fontenc}

\usepackage{palatino}
\usepackage[pdftex]{hyperref}
\usepackage{pgf}
\usepackage[final]{svninfo}

\usepackage{amsmath}
\usepackage{amsfonts}

%Definitions of numbers
\def\Aset{\mathbb{A}}
\def\Cset{\mathbb{C}}
\def\Kset{\mathbb{K}}
\def\Nset{\mathbb{N}}
\def\Pset{\mathbb{P}}
\def\Rset{\mathbb{R}}
\def\Sset{\mathbb{S}}
\def\Tset{\mathbb{T}}
\def\Xset{\mathbb{X}}
\def\Yset{\mathbb{Y}}
\def\Zset{\mathbb{Z}}

%----------------------------------------------------------------------
% Listados de código Octave, Python mediante el paquete lstlistings
%----------------------------------------------------------------------

\usepackage{color}
\definecolor{gray96}{gray}{.96}
\definecolor{gray70}{gray}{.70}
\definecolor{gray40}{gray}{.40}
\definecolor{gray25}{gray}{.25}

\usepackage{listings}
\lstset{ frame=Ltb,
     framerule=0pt,
     aboveskip=0.5cm,
     framextopmargin=3pt,
     framexbottommargin=3pt,
%     framexleftmargin=1cm,
     framesep=0pt,
     rulesep=.4pt,
     backgroundcolor=\color{gray96},
     rulesepcolor=\color{black},
     %
     stringstyle=\ttfamily,
     showstringspaces = false,
     basicstyle=\small\ttfamily,
     commentstyle=\color{gray40},
     keywordstyle=\bfseries,
     identifierstyle=\itshape\bfseries\color{gray25},
     %
     numbers=left,
     numbersep=15pt,
     numberstyle=\tiny,
     numberfirstline = false,
     breaklines=true,
   }

% minimizar fragmentado de listados
\lstnewenvironment{listing}[1][]
   {\lstset{#1}\pagebreak[0]}{\pagebreak[0]}

\lstdefinestyle{octave}
   {language=octave,
   }

\lstdefinestyle{python}
   {language=python,
   }

% No disponible maxima :-(
% \lstdefinestyle{maxima}
%    {language=maxima,
%    }

%----------------------------------------------------------------------
% Definición de \etiqueta{a}, \boton{a}, \menu{a,b,c},...
%----------------------------------------------------------------------
\newcommand{\etiqueta}[1]{\textsf{#1}}
\newcommand{\boton}[1]{[\textsf{#1}]}
\def\menu#1{%
  \def\@separador{}
  \COMAS@do#1,\relax,
  \def\@separador{}}
\def\COMAS@do#1,{%
  \ifx\relax#1\empty\else
  \@separador''\etiqueta{#1}''\def\@separador{ $\!\rightarrow\!$ }%
  \expandafter\COMAS@do\fi}

%
% Defino \imagen como interfaz hacia \pgifmage y \captura hacia figure
%
\newcommand{\imgpath}{img}
\newcommand{\imagen}[2][0.75\linewidth]{%  
  \pgfimage[width=#1]{\imgpath/#2}
}
\newenvironment{captura}[3][0.75\linewidth]{%
    \begin{figure}
    \center
    \fbox{\imagen[#1]{#2}}
    \caption{#3}}
  {\end{figure}}

\newcommand{\tecla}[1]{\textsf{#1}}

%----------------------------------------------------------------------
% Definiciones específicas para octave
%----------------------------------------------------------------------

\newenvironment{octavesession}{
  \renewcommand{\i}{Octave }
  \renewcommand{\o}{}
}{}


\newcommand{\en}{\mbox{\ en \ }}
\newcommand{\sobre}{\mbox{\ sobre \ }}

\title{Práctica de diferencias finitas con Python}

\begin{document}

  En este documento utilizaremos Python para  la
  aproximación numérica de ecuaciones en derivadas parciales mediante
  el método de las diferencias finitas.

El método de las diferenciasfinitas consiste en reemplazar cada una de
las derivadas parciales por aproximaciones mediante cocientes
incrementales de orden uno o dos. Entre sus ventajas, se encuentra el
tratarse de un método intuitivo y fácil de implementar. Entre sus
inconvenientes, no es sencilla su generalización a dominios distintos
de intervaloso, en general, a cubos $n-$dimensionales ($n=1,2,...$).

\section{Diferencias finitas 1D}
Consideremos el siguiente problema diferencial de orden 2:
Calcular $u:[a,b] \to  \Rset$ solución de
\begin{align}
  \label{pb1d}
  -\frac{d^2 u(x) }{dx^2} = f(x) \en [a,b], \\
  u(a)=u_a, \ u(b)=u_b,
\end{align}
donde tanto $f$ como los datos de contorno, $u_a$ y $u_b$ son
previamente conocidos. Bajo unas hipótesis mínimas de regularidad para
$f$ (bastaría $f\in L^2(a,b)$, pero para utilizar el método de las
diferencias finitas supondremos que $f$ es contínua), se sabe que el
problema de contorno anterior tiene una única solución.

\subsection{Aproximación mediante el método de diferencias finitas}
\label{sec:diferencias-finitas-dim-1}
Consideremos una partición del intervalo $[a,b]$ en $n+1$
subintervalos, todos ellos de longitud $h=(b-a)/(n+1)$:

$$
a=x_0 < x_1 < \cdots < x_{n+1} = b.
$$

Calcularemos, en cada uno de los puntos $x_i$, una aproximación del
valor de $u(x_i)$, a la que llamaremos $u_i$. Conocemos los valores $u_0$ y
$u_{n+1}$ (pues deben coincidir, respectivamente, con los datos $u_a$
y $u_b$), de forma que nuestras incógnitas son, exactamente, $u_1,
u_2,\dots,u_n$. Para calcularlas, realizamos, en el sistema anterior,
una aproximación de segundo orden\footnote{Que no es difícil de
  deducir, utilizando los desarrollos de Taylor de $u$ en
$x-h$ y en $x+h$. Se puede demostrar, de hecho, que si $u \in
C^4([a,b])$, el esquema propuesto es  consistente
y convergente (de orden $2$ en $h$).} de la derivada segunda 
:
$$
u''(x) \approx \frac{u(x-h)-2u(x)+u(x+h)}{h^2}.
$$

Utilizando esta fórumula en (\ref{pb1d}) para los puntos $x_i$
($i=1,\dots,n$), obtenemos el siguiente sistema de ecuaciones:
$$
-\frac{u_{i-1}-2u_i + u_{i+1}}{h^2} = f_i, \quad i=1,\dots,n.
$$
siendo $f_i=f(x_i)$.
Si las observamos cuidadosamente, concluiremos que forman un sistema
lineal como el que sigue:

\begin{gather}
Au_h=f,\label{eq:sistema1D} \\
\intertext{para la matriz}
A=\frac{1}{h^2}\left(
\begin{array}{rrrrrr}
   2 & -1 &  0 &  0 & \dots & 0 \\
  -1 &  2 & -1 &  0 & \dots & 0 \\
   0 & -1 &  2 & -1 & \dots & 0 \\
     &    & \ddots & \ddots & \ddots \\
   0 & \dots & & -1 & 2  & -1 \\
   0 & \dots & & 0 & -1 & 2  
 \end{array}
\right),
\\
\intertext{y los vectores}
u_h=
\begin{pmatrix}
  u_1 \\ u_2 \\ \vdots \\ u_{n-1} \\ u_n
\end{pmatrix}
%
\quad \mbox{y}  \quad
f=
\begin{pmatrix}
  f_1 + u_a/h^2 \\ f_2 \\ \vdots \\ f_{n-1} \\ f_n+u_b/h^2
\end{pmatrix}
\end{gather}
La matriz $A$ es tridiagonal y definida positiva, lo que dota al
sitema lieal annterior de muy buenas propiedades de cara a su
implementación y resolución, especialmente pensando en que $n$ sea muy
grande.

\subsection{Resultados numéricos con Python}
Para construir el sistema (\ref{eq:sistema1D}), podemos utilizar la
función ``diag''.

Se propone resolver la ecuación (\ref{pb1d}) para los siguientes
parámetros:
\begin{itemize}
\item $[a,b]=[0,1]$
\item $n= 10$;
\item $h = 1/(n+1)$;
\item Datos de contorno: $u_a=0$; $u_b=1$;
\item Segundo miembro: $f(x)=\frac{\pi^2}{4}  \sin(x\pi/2)$
\end{itemize}

\textbf{Extensión}: definir una función, \verb|diferencias_finitas_1d|,
que tome como parámetros:
\begin{itemize}
\item Los extremos del intervalo ($a$, $b$) y el tamaño de la
  partición ($n$).
\item Los datos de contorno ($u_a$ y $u_b$) y la función segundo
  miembro, $f$.
\end{itemize}
La función debe devolver el vector solución, $u$.

\section{Diferencias finitas 2D}

\subsection{Aproximación mediante el método de diferencias finitas}
\label{sec:diferencias-finitas-dim-2}

Consideremos un dominio rectangular, $\Omega=(a_x,b_x)\times(a_y,b_y)
\subset \Rset^2$. 
Planteamos el problema: Calcular $u:\Omega \to  \Rset$ solución de
\begin{align}
  \label{pb1d}
  -\frac{\partial^2 u(x,y) }{\partial x^2} 
  -\frac{\partial^2 u(x,y) }{\partial y^2} 
  = f(x,y) \en \Omega, \\
  u=g \sobre \partial\Omega
\end{align}
donde $\partial\Omega$ es la frontera de $\Omega$ y tanto $f(x,y)$
como $g(x,y)$ son funciones conocidas, que supondremos continuas.


\subsection{Aproximación mediante el método de diferencias finitas}
\label{sec:diferencias-finitas-dim-2}

Definiremos un mallado uniforme del rectángulo $\Omega$, a través de
una partición de $[a_x,b_x]$ con talla $h_x = (b_x-a_x)/(N_x+1)$ y de
una partición de $[a_y,b_y]$ con talla $h_y = (b_y-a_h)/(N_y+1)$.

Estas particiones definen los puntos $(x_i,y_j)\in\Rset^2$, donde
\begin{align*}
  x_i &= i\cdot h_x, i=0,1,...,N_x+1
  \\
  y_j &= j\cdot h_y, j=0,1,...,N_y+1
\end{align*}

Calcularemos, en cada uno de los puntos $(x_i,y_j)$, una aproximación del
valor de $u(x_i,y_j)$, a la que llamaremos $u_{i,j}$.

De forma similar al caso $1D$, podemos aproximar cada una de las
derivadas parciales. Sumándolas, obtendremos la siguiente aproximación
del Laplaciano:
\begin{align*}
  \frac{\partial^2 u(x,y) }{\partial x^2}
  +
  \frac{\partial^2 u(x,y)}{\partial y^2} 
  & \approx \frac{u(x-h_x,y)-2u(x,y)+u(x+h_x,y)}{h_x^2} 
  \\
  &+
  \frac{u(x,y-h_y)-2u(x,y)+u(x,y+h_y)}{h_y^2}
\end{align*}

Utilizando esta fórmula en los puntos $(x_i,y_j)$ de la malla,
obtenemos el siguiente sistema de ecuaciones:
$$
-\frac{u_{i-1,j}-2u_{i,j} + u_{i+1,j}}{h_x^2} 
-\frac{u_{i,j-1}-2u_{i,j} + u_{i,j+1}}{h_y^2} 
= f_{i,j},
$$
siendo $f_{i,j}=f(x_{i},y_j), \  \quad i=1,\dots,N_x,  \quad i=1,\dots,N_y.$.

Para transformar las ecuaciones anteriores en un sistema lineal,
debemos escribir nuestras incógnitas, $u_{i,j}$ en forma de
vector. Para ello, basta realizar una simple renumeración de los
índices, por ejemplo, podemos definir el vector:
$$
u=[(u_{1,1}, u_{2,1},...,u_{N_x,1}), (u_{1,2}, u_{2,2},...,u_{N_x,2}),
...
(u_{1,N_y}, u_{2,N_y},...,u_{N_x,N_y})
$$


Si las observamos cuidadosamente, concluiremos que forman un sistema
lineal, $Au=b$, donde la matriz viene dada por:
$$
A=
\begin{pmatrix}
   a & -b &  0 &  0 & \dots & -c & 0 & \dots & 0 \\
  -b &  a & -b &  0 & \dots & 0 & -c & \dots & 0\\
   0 & -b &  a & -b & \dots & 0 & 0 & \ddots & 0 \\
   \vdots &    & \ddots & \ddots & \ddots & \\
   -c & 0 & \dots & -b & a  & -b \\
   0 & -c &  0 & \dots & -b & a  & -b \\
     & & &&    & \ddots & \ddots & \ddots & \\
   0 & \dots & &  & & & 0 & -b & a  
 \end{pmatrix},
$$
siendo: 
\begin{align*}
  a&=\frac{2}{h_x^2}+\frac{2}{h_y^2} \\
  b&=\frac{1}{h_x^2} \\
  c&=\frac{1}{h_y^2}
\end{align*}
El vector diagonal $(c,c,\dots,c)$ está separado exactamente $N_x$
posiciones desde la diagonal de la matriz.

En cuanto al segundo miembro, incluirá a los valores
$f_{i,j}=f(x_i,y_j)$ (a través de una renumeración idéntica a la
realizada para el vector $u$) y a los términos frontera
$(1/h_x^2)g_{i,j}$ y $(1/h_y^2)g_{i,j}$, en las posiciones adecuadas.
Para simplificar el problema, podemos asumir que la condición de
contorno $g$, es nula salvo en el lado del rectángulo correspondiente
a los puntos $(x_0,y_0), (x_1,y_0),...,(x_{N_x+1},y_0)$.

En tal caso, debemos sumar a los $N_x$ primeros términos segundo
miembro los valores:
\begin{align*}
  \frac{1}{h_y^2} g_{i,0}, \quad i=1,...,N_x 
\end{align*}


\subsection{Resultados numéricos con Python}
Se propone resolver la ecuación (\ref{pb1d}) para los siguientes
parámetros:
\begin{itemize}
\item $\Omega=[0,1]^2$
\item $N_x=N_y=10$;
\item $f=0$, $g=1$ en los puntos $(x_0,y_0), (x_1,y_0),...,(x_{N_x+1},y_0)$.
\end{itemize}



\end{document}
%%% Local Variables: 
%%% mode: latex
%%% TeX-master: t
%%% End: 
